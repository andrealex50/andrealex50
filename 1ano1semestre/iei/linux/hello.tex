\documentclass[11pt,openright,twoside]{report}
\usepackage[utf8]{inputenc}
\usepackage[T1]{fontenc}
\usepackage[portuguese]{babel}
\usepackage{showlabels}

\title{\textbf{Introdução à Engenharia Informática}}
\author{Turma 1 Grupo 1}
\date{}

\begin{document}

\renewcommand{\tableofcontents}{BLA BLA BLA}
\maketitle
Hello World from \LaTeX!


\tableofcontents


Ora boas pessoal.
Hoje joga o fê quê pê.
Olha o texto a aumentar de \large tamanho.
\renewcommand{\tableofcontents}{Índice}

Uma lista não numerada é:
\begin{itemize}
    \item este é um item
    \item este é outro item
    \item etc.
\end{itemize}
Aqui retoma a indentação

\begin{itemize}
    \item[vermelho] - é mau
    \item[azul] - é bom
    \item[verde] - é bom
\end{itemize}


\begin{table}[htp]
\caption{Exemplo de uma tabela}
    \centerline{Conteúdo de uma tabela}
\label{tabela-exemplo}
\end{table}%

\begin{figure}[hb]
    \centerline{\fbox{Conteúdo da figura, pode ser texto, imagem, etc.}}
\caption{Legenda da figura}
\label{figura-exemplo}
\end{figure}

\begin{figure}[h]
    \centerline{\fbox{Outra figura}}
\caption{Legenda de outra figura}
\label{figura1}
\end{figure}

Olha que figura linda \ref{figura-exemplo}


\begin{tabular}{|l||c|r|} 
%
\hline
     & Temperatura & Humidade \\
Cidade & (\textordmasculine C) & (perc.) \\ \hline\hline
Aveiro & {\large 10} & {\large 90} \\ \hline
Lisboa & {\tiny 13} & {\tiny 84} \\ \hline
Porto & \textbf{9} & \textbf{89} \\ \hline
%
\end{tabular}   

\newpage
a expressão $$ = ax^2 + bx + c$$ é a forma geral de equação de 2ºgrau

a expressão $y = ax^2 + bx + c$ é a forma geral da equação
de 2º grau

$y = \sum_{i=0}^{i=n}{x_{i}^{2n}}$
$$y = \sum_{i=0}^{i=n}{x_{i}^{2n}}$$





\begin{figure}[h]
    \center
    \includegraphics[height=24pt]{gaivota}
    \caption{Gaivota}
    \label{fig:gaivota.png}
\end{figure}
    

\end{document}
